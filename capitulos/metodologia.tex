\section{METODOLOGIA}
\textcolor{red}{(apresentar 1 linha livre)}

Este texto já se encontra no padrão de espaçamento correto. Ao elaborar o seu conteúdo, verificar se o espaçamento entre linhas é de 1,5 e se há uma linha livre entre parágrafos com espaçamento 0 pt antes e depois. 
\cite{reforma}

Esclarecer se a pesquisa é de natureza básica ou aplicada e, quanto aos objetivos, se é exploratória, descritiva ou explicativa. Indicar também o procedimento a ser adotado: pesquisa experimental, levantamento, estudo de caso, pesquisa bibliográfica, ou outro. 

Definir o universo de estudo e os critérios de inclusão e exclusão do processo de amostragem. 

Descrever as técnicas utilizadas para a coleta de dados e os instrumentos utilizados (de acordo com o tipo de técnica escolhida), a serem apresentados em anexo, se necessário. 

A coleta de dados é a busca por informações para a elucidação do fenômeno ou fato que o pesquisador quer desvendar. O instrumental técnico elaborado pelo pesquisador para o registro e a medição dos dados deverá preencher os seguintes requisitos: validez, confiabilidade e precisão.  

Descrever a metodologia de análise de dados e as ferramentas utilizadas para tal fim. 
