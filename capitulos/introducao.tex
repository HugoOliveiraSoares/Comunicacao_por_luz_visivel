\section{Introdução}

Com o aumento da popularidade da internet em todo o mundo, é notável que as redes \textit{wifi} têm crescido significativamente, juntamente com o número de usuários e de dispositivos IoT (\textit{Internet Of Things}) \cite{matheus2017comunicaccao}. De acordo com o relatório Digital 2023: Global Overview Report \cite{datareportal}, publicado pelo site Datareportal, há cerca de 5,16 bilhões de usuários na internet. No entanto, esse aumento na demanda por \textit{wifi} tem causado um problema, que é a congestão das faixas do espectro eletromagnético reservadas para essas redes, assim afetando a sua eficiência.\newline

\begin{figure}[!htbp]
  \caption{Indicadores de uso da Internet}
  \includegraphics[scale=0.4]{images/internet_use.png}
  \legend{Fonte: \citeonline{datareportal}}
  \label{figura:usoInternet}
\end{figure}

As redes \textit{wireless} utilizam ondas eletromagnéticas para a transmissão de dados e informações, o que inviabiliza ou dificulta a sua utilização em alguns lugares, como em hospitais e aeronaves, por exemplo, por interferir com equipamentos hospitalares e com a antena de transmissão no caso dos voos.

Diante desses cenários, o \textit{Visible Light Communication} (VLC) se mostra como um forte candidato para a solução destes problemas.
Verifica-se que o espectro da luz visível, possui 10 mil vezes mais faixas de frequência se comparado com as ondas de rádio \cite[p. 14]{conceiccao2015comunicaccao}.
Ou seja, é possível que um único roteador se comunique com mais dispositivos ao mesmo tempo.

Para o problema de interferência o VLC também é uma solução, visto que utiliza a luz visível como forma de transmitir as informações, assim não gerando interferências eletromagnéticas em outros aparelhos eletrônicos ou em redes \textit{wifi}.

O estudo objetiva verificar a viabilidade de implementação do sistema VLC com um \textit{SBC (Single Board Computer)}, através da construção de um protótipo. A pesquisa experimental surgiu da necessidade de uma nova forma de transmissão de dados com pouca interferência e de baixo custo, abrindo uma  possibilidade de levar comunicação em locais onde não era possível recorrer a uma rede \textit{wireless}.

