\chapter{Justificativa}

Segundo o autor \citeauthoronline{tanenbaum} o comitê do IEEE definiu que as redes no padrão 802.11, \textit{wifi}, utilizariam as frequências de 2,4GHz e 5GHz e que todos os dispositivos têm a permissão para utilizá-los desde que limitem a sua potência para permitir que dispositivos diferentes coexistam. O autor \citeonline{barros} também cita que há muitos outros equipamentos eletrônicos que geram ondas também na faixa de 2,4GHz. 

Devido a faixa de 2,4GHz ser internacionalmente regulamentada ela não necessita de licença para a sua utilização e com a popularidade das redes sem fio a faixa tem concentrado grande parte da demanda por frequência. Assim o seu compartilhamento tem se tornado bastante denso fazendo com que os receptores lidem constantemente com interferências \cite{barros}. Segundo \citeonline{fcc} o problema do congestionamento do espectro é crescente e está cada vez mais comum nas residências.

Outro problema enfrentado pelas redes sem fio é a interferência eletromagnética gerada por dispositivos elétricos que pode afetar o funcionamento da comunicação e vice-versa, como por exemplo um forno de microondas. Esses aparelhos operam na faixa de 2,45GHz, assim provocando um aumento nas taxas de erro nos dados que trafegam nas redes \cite{barros}. Já no caso contrário as redes móveis podem provocar alterações no funcionamento de dispositivos hospitalares e colocar em risco a vida dos pacientes \cite{cabral}.

A construção de um protótipo de um sistema VLC se dá pelo seu grande potencial de solucionar os problemas citados acima. Como o VLC utiliza uma faixa de comprimento de onda que vai de cerca de 380nm até 780nm, permite que a tecnologia ofereça uma faixa de frequências cerca de 10 mil vezes maior do que a radiofrequência, permitindo que mais dispositivos se conectem no mesmo ponto de acesso assim solucionando o problema do congestionamento do espectro \cite{conceiccao2015comunicaccao}. 

Como a luz visível não interfere em equipamentos eletrônicos, o VLC tem a possibilidade de operar  em locais onde a RF não é desejada, como por exemplo em hospitais, evitando o mau funcionamento dos dispositivos.

