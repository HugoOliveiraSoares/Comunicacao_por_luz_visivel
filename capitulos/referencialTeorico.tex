\section{Referencial teórico}  

\subsection{Padrão IEEE 802.11}

Quando os computadores receberam transmissores e receptores de rádio varias empresas começaram a comercializar LANs sem fios, porém não havia uma padronização para a comunicação, ou seja, um computador equipado com um rádio da marca \emph{X} não era compatível com o computador equipado com o rádio da marca \emph{Y}. Diante deste problema surgiu a necessidade de se criar um padrão para as LANs sem fios, assim o comitê do IEEE criou o padrão 802.11 mais conhecido como \textit{wifi} \cite{tanenbaum}.

\subsubsection{Faixas 2.4Ghz e 5Ghz}

As faixas de radio que o \textit{wifi} utiliza são as faixas de 2,4GHz e 5GHz, as duas bandas não necessitam de licença para a sua utilização contudo os aparelhos devem limitar a sua potência para permitir que diferentes dispositivos coexistam \cite{tanenbaum}. Como a utilização da faixa é livre é muito provável que os equipamentos de \textit{wifi} tenham que lidar constantemente com interferências. 

\subsection{Interferência eletromagnética}

\subsection{OpenVLC}
