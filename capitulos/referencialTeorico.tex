\section{REFERENCIAL TEÓRICO}  

\textcolor{red}{(apresentar 1 linha livre)}

Este texto já se encontra no padrão de espaçamento correto. Ao elaborar o seu conteúdo, verificar se o espaçamento entre linhas é de 1,5 e se há uma linha livre entre parágrafos com espaçamento 0 pt antes e depois. 

Expor resumidamente as principais ideias já discutidas por outros autores que trataram do problema, levantando críticas e dúvidas, quando for o caso. Explicar no que seu trabalho vai se diferenciar dos trabalhos já produzidos sobre o problema a ser trabalhado e/ou no que vai contribuir para seu conhecimento. Quanto ao quadro teórico, o erro mais frequente é formulá-lo de forma genérica ou abstrata demais, quando o que interessa é que ele seja adequado ao recorte temático a ser investigado; quanto à formulação das hipóteses ou das questões, não basta enunciá-las no projeto, é preciso também justificá-las uma a uma em texto argumentativo. 

\noindent
\textcolor{red}{FIGURAS E FOTOS - Padrão: Júnia Lessa França (2019) 10ª ed. – pág 106 
\\
GRÁFICOS - Padrão: Júnia Lessa França (2019) 10ª ed. – pág 111 
\\
TABELAS E QUADROS - Padrão: Júnia Lessa França (2019) 10ª ed. – pág 111 }
