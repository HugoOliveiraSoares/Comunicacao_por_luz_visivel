\section{Conclusão}

O objetivo desta pesquisa centrava-se em construir um protótipo de um sistema que se comunica usando a tecnologia do VLC, ou seja, permite que dois dispositivos se comuniquem usando a luz visível. Como o VLC opera no espectro da luz visível ele não interfere em equipamentos eletrônicos e também não é afetado por eles, podendo ser utilizado em locais onde ondas de rádio não são a melhor solução \cite{matheus2017comunicaccao}.

Com o intuito de atingir este objetivo foi necessário revisar o trabalho de outros autores e colocar em prática tudo aquilo que foi estudado. Para isso construiu-se um \textit{software} capaz de modular 5 \textit{bytes} em luz e outro capaz de demodular a informação em forma de luz em \textit{bytes} novamente, obtendo uma comunicação bem sucedida e com taxa de transmissão de 10 \textit{bits} por segundo.

Durante o desenvolvimento enfrentou-se diversos pontos que dificultaram a construção do protótipo, como o projeto da SBC \textit{OrangePi} não ser muito maduro não possuindo muita documentação sobre a sua biblioteca para manipular as GPIOs. Outro ponto é o uso de um circuito muito simples e de baixa responsividade a variação de luz, para receber o sinal.

Diante desses pontos, pode-se concluir que para a comunicação alcançar maiores taxas de transferência a implementação de um \textit{hardware} específico para a modulação e demodulação do sinal representaria um avanço significativo. A utilização de um microcontrolador dedicado a essas funções, comunicando-se via serial com a SBC, promoveria uma transmissão mais eficiente.

Ao alcançar o objetivo de construir o protótipo, demonstra a viabilidade da existência da tecnologia abrindo uma perspectiva promissora para solucionar os problemas de congestionamento do espectro eletromagnético e da interferência gerada por dispositivos elétricos. 

Algumas restrições foram fundamentais para reduzir a amplitude dos resultados, embora aqueles alcançados tenham sido expressivos. Uma das principais restrições foi o acesso limitado a \textit{hardware} mais robusto, englobando não apenas as placas mais recentes da linha Raspberry, isso se deve a escassez global de semicondutores, o que reduziu significativamente sua disponibilidade e consequentemente levou a um aumento nos preços \cite{zeng_2022}. E também a dificuldade de adiquirir o \textit{hardware} desenvolvido especificamente pelo projeto OpenVLC, por causa da restrição de tempo disponível para a execução do projeto limitado a parte do período letivo do segundo semestre de 2023.
