\renewcommand{\imprimircapa}{
  \begin{capa}
    \center

    \begin{center}
      {\ABNTEXchapterfont\bfseries\LARGE\imprimirtitulo}
    \end{center}

    \vspace*{1cm}

    {\ABNTEXchapterfont\large\imprimirautor\footnote{Graduando em Ciência da Computação, Escola Superior Dom Helder, \href{mailto:e01381@ academico.domhelder.edu.br}{e01381@academico.domhelder.edu.br}}}
    \\
    {\ABNTEXchapterfont\large\imprimirorientador\footnote{Mestre em Tecnologia da informação(CEFET), docente do curso Ciência da Computação, Escola Superior Dom Helder Câmara, marden.cicarelli@academico.domhelder.edu.br}}
    \\
    {\ABNTEXchapterfont\large\imprimircoorientador\footnote{Mestre em Administração (FUMEC), docente do curso de Ciência da Computação, Escola Superior Dom Helder, \href{mailto:ricardo.freitas@academico.domhelder.edu.br}{ricardo.freitas@academico.domhelder.edu.br}}}

     \vspace{\onelineskip}

     % RESUMO
     \begin{flushleft}
      {\ABNTEXchapterfont\bfseries\large Resumo}

      O Visible Light Communication (VLC) é uma tecnologia que utiliza o espectro da luz visível para a transmissão de dados. Com o aumento da popularidade da internet e do uso de dispositivos IoT, a demanda por redes wifi tem crescido exponencialmente, causando congestão nas faixas de espectro eletromagnético destinadas a essas redes. O VLC surge como uma solução promissora para este problema, aproveitando as diversas faixas de frequência disponíveis na luz visível. Além disso, o VLC evita interferências eletromagnéticas em dispositivos eletrônicos e redes wifi. O estudo demonstrou a viabilidade da implementação do sistema VLC com a Single Board Computer (SBC) OrangePi, tendo como resultado um protótipo bem sucedido em realizar uma transmissão de dados.
      \vspace{\onelineskip}
 
      \noindent
      \textbf{Palavras-chave}: VLC. SBC. OpenVLC. RaspberryPi. OrangePi
    \end{flushleft}

    \vspace{\onelineskip}
    \footnoterule

  \end{capa}
}


