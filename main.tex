\documentclass[12pt,a4paper,english,brazil]{abntex2}

\usepackage[brazil]{babel}
\usepackage[left=3.0cm,top=3.0cm,right=2.0cm,bottom=2.0cm]{geometry}
\usepackage{setspace}
\usepackage{indentfirst}
\usepackage{graphicx}
\usepackage{xcolor}
\usepackage{microtype} 			% para melhorias de justificação
\usepackage[alf]{abntex2cite}	% CITAÇÕES PADRÃO ABNT
\usepackage{fontspec}
\usepackage{url}
\setmainfont{Arial}

% O tamanho do parágrafo(recuo) é dado por:
\setlength{\parindent}{1.25cm}
\linespread{1.5}
% \OnehalfSpacing

\selectlanguage{brazil}

\makeatletter
\hypersetup{
      pdftitle={\@title}, 
		  pdfauthor={\@author},
	    pdfcreator={Hugo Soares},
      pdfkeywords={VLC}{Visible Light Communication}{Raspberripy}{Dom Helder}{OpenVLC},
  		colorlinks=true,
    	citecolor=black,
      linkcolor=black,
      urlcolor=blue,
	    bookmarksdepth=4
}

\addto\captionsbrazil{
%% ajusta nomes padroes do babel
\renewcommand{\bibname}{Referências}
\renewcommand{\indexname}{Indice}
\renewcommand{\listfigurename}{Lista de ilustrações}
\renewcommand{\listtablename}{Lista de tabelas}
%% ajusta nomes usados com a macro \autoref
\renewcommand{\pageautorefname}{pagina}
\renewcommand{\sectionautorefname}{seção}
\renewcommand{\subsectionautorefname}{subseção}
\renewcommand{\paragraphautorefname}{parágrafo}
\renewcommand{\subsubsectionautorefname}{subseção}
}

\data{2023}

\begin{document}
  
  \titulo{Comunicação por luz visível: \\ Construção de um protótipo com Raspberry Pi para explorar o potencial da tecnologia VLC}
  \autor{Hugo Oliveira Soares}
  \local{Belo Horizonte}
  \instituicao{Dom Helder Escola Superior}
  \orientador{Prof. Marden Cicarelli Pinheiro}
  \coorientador{Prof. Ricardo Luiz de Freitas}
  \preambulo{
    Projeto de Pesquisa apresentado à Dom Helder Escola Superior como requisito parcial para obtenção do título de Cientista da Computação.
    \newline 
    \newline
    Orientador de conteúdo: \imprimirorientador
    \newline
    \newline
    Orientador de metodologia: \imprimircoorientador
  }

  \pagenumbering{roman}
  % PARTE PRÉ-TEXTUAL
  \renewcommand{\imprimircapa}{
  \begin{capa}
    \center

    \begin{figure}[ht]
        \begin{flushright}
          \includegraphics[scale=0.8]{images/logoDom.png}
          \label{Logo_DomHelder}
        \end{flushright}
    \end{figure}
    \vspace*{0.5cm}

    {\ABNTEXchapterfont\Large\imprimirinstituicao}

    \vspace*{4cm}

    {\ABNTEXchapterfont\large\imprimirautor}

    \vspace*{3cm}
    \begin{center}
      {\ABNTEXchapterfont\bfseries\LARGE\imprimirtitulo}
    \end{center}
    \vfill

   
    \imprimirlocal
    \par
    \imprimirdata

    \vspace*{1cm}
  \end{capa}
}


 
  \include{folharosto}
  \imprimircapa
  \imprimirfolhaderosto

  \listoffigures
\cleardoublepage
  \begin{siglas}

  \item[IOT]  Internet Of Things
  \item[VLC]  Visible Light Communication  

\end{siglas}

  \newpage
  \tableofcontents %SUMARIO

  \pagenumbering{arabic}
  % \textual
  %% CAPITULOS
  \chapter{Introdução}

Com o aumento da popularidade da internet em todo o mundo, é notável que as redes WiFi têm crescido significativamente, juntamente com o número de usuários e de dispositivos IoT (Internet Of Things). De acordo com o relatório Digital 2023: Global Overview Report, publicado pelo site Datareportal, há cerca de 5,16 bilhões de usuários da internet atualmente. No entanto, esse aumento na demanda por WiFi tem causado um problema, que é a congestão das faixas do espectro eletromagnético reservadas para essas redes, assim afetando a sua eficiência.

\begin{figure}[htb]
  \centering
  \caption{Indicadores de uso da Internet}
  \includegraphics[scale=0.45]{images/internet_use.png}
  \legend{Fonte: \citeonline{Datareportal}}
  \label{figura:usoInternet}
\end{figure}

As redes  wireless utilizam ondas eletromagnéticas para a transmissão de dados e informações, o que inviabiliza ou dificulta a sua utilização em alguns lugares, como em hospitais e aeronaves, por exemplo, por interferir com equipamentos hospitalares e com a antena de transmissão no caso dos voos.

Se deparando com esses cenários, o Visible Light Communication (VLC) se mostra como um forte candidato para a solução destes problemas. Pois percebemos que o espectro da luz visível, possui 10 mil vezes mais faixas de frequência se comparado com as ondas de rádio, \citeonline[p. 14]{conceiccao2015comunicaccao}. Ou seja, é possível que um único “roteador” se comunique com mais dispositivos ao mesmo tempo.

Para o problema de interferência o VLC também é uma solução, visto que utiliza a luz visível como forma de transmitir as informações, assim não gerando interferências eletromagnéticas em outros aparelhos eletrônicos ou em redes wifi.

Assim, o estudo objetiva verificar a viabilidade de implementação do sistema VLC com o mini-computador Raspberry Pi, através da construção de um protótipo. A pesquisa experimental surgiu da necessidade de uma nova forma de transmissão de dados com pouca interferência e de baixo custo. Abrindo uma  possibilidade de levar comunicação em locais onde não era possível recorrer a uma rede wireless.


  % \include{capitulos/hipoteses}
  \chapter{Objetivos} \label{obj}
\section{Objetivo geral}

O propósito desta pesquisa é a construção de um protótipo de um sistema de
comunicação VLC, baseado no projeto \textit{OpenVLC}, utilizando exemplares de SBC. O objetivo principal é que o sistema seja capaz de transmitir e receber um pequeno pacote de dados.

\section{Objetivos específicos}

\begin{itemize}

  \item Explicar o que é VLC
    \begin{itemize}
      \item Explicar o funcionamento
      \item Analisar as vantagens, desvantagens e desafios.
    \end{itemize}

  \item Implementar um protótipo
  \item Avaliar o desempenho do protótipo
    \begin{itemize}
      \item Comparar o desempenho entre as SBCs selecionadas
    \end{itemize}

\end{itemize}


  \chapter{Justificativa}

Segundo o autor \citeauthoronline{tanenbaum} o comitê do IEEE definiu que as redes no padrão 802.11, \textit{wifi}, utilizariam as frequências de 2,4GHz e 5GHz e que todos os dispositivos têm a permissão para utilizá-los desde que limitem a sua potência para permitir que dispositivos diferentes coexistam. O autor \citeonline{barros} também cita que há muitos outros equipamentos eletrônicos que geram ondas também na faixa de 2,4GHz. 

Devido a faixa de 2,4GHz ser internacionalmente regulamentada ela não necessita de licença para a sua utilização e com a popularidade das redes sem fio a faixa tem concentrado grande parte da demanda por frequência. Assim o seu compartilhamento tem se tornado bastante denso fazendo com que os receptores lidem constantemente com interferências \cite{barros}. Segundo \citeonline{fcc} o problema do congestionamento do espectro é crescente e está cada vez mais comum nas residências.

Outro problema enfrentado pelas redes sem fio é a interferência eletromagnética gerada por dispositivos elétricos que pode afetar o funcionamento da comunicação e vice-versa, como por exemplo um forno de microondas. Esses aparelhos operam na faixa de 2,45GHz, assim provocando um aumento nas taxas de erro nos dados que trafegam nas redes \cite{barros}. Já no caso contrário as redes móveis podem provocar alterações no funcionamento de dispositivos hospitalares e colocar em risco a vida dos pacientes \cite{cabral}.

A construção de um protótipo de um sistema VLC se dá pelo seu grande potencial de solucionar os problemas citados acima. Como o VLC utiliza uma faixa de comprimento de onda que vai de cerca de 380nm até 780nm, permite que a tecnologia ofereça uma faixa de frequências cerca de 10 mil vezes maior do que a radiofrequência, permitindo que mais dispositivos se conectem no mesmo ponto de acesso assim solucionando o problema do congestionamento do espectro \cite{conceiccao2015comunicaccao}. 

Como a luz visível não interfere em equipamentos eletrônicos, o VLC tem a possibilidade de operar  em locais onde a RF não é desejada, como por exemplo em hospitais, evitando o mau funcionamento dos dispositivos.


  % \section{Referencial teórico}  

\subsection{Padrão IEEE 802.11}

Quando os computadores receberam transmissores e receptores de rádio varias empresas começaram a comercializar LANs sem fios, porém não havia uma padronização para a comunicação, ou seja, um computador equipado com um rádio da marca \emph{X} não era compatível com o computador equipado com o rádio da marca \emph{Y}. Diante deste problema surgiu a necessidade de se criar um padrão para as LANs sem fios, assim o comitê do IEEE criou o padrão 802.11 mais conhecido como \textit{wifi} \cite{tanenbaum}.

\subsubsection{Faixas 2.4Ghz e 5Ghz}

As faixas de radio que o \textit{wifi} utiliza são as faixas de 2,4GHz e 5GHz, as duas bandas não necessitam de licença para a sua utilização contudo os aparelhos devem limitar a sua potência para permitir que diferentes dispositivos coexistam \cite{tanenbaum}. Como a utilização da faixa é livre é muito provável que os equipamentos de \textit{wifi} tenham que lidar constantemente com interferências. 

\subsection{Interferência eletromagnética}

\subsection{OpenVLC}

  % \section{Metodologia}

Segundo \citeonline{wiltgen2019prototipos}, um protótipo é uma representação similar ao produto a ser desenvolvido, criado com o intuito de realizar testes e ensaios para que as funcionalidades se comportem como esperado no ambiente de uso. Nesse contexto, o objetivo principal deste trabalho é implementar um protótipo de um sistema VLC utilizando um SBC como plataforma de desenvolvimento. Durante o desenvolvimento foram utilizados elementos de metodologias ágeis como os do \textit{Scrum} e do \textit{Kanban}.

O \textit{Scrum} é uma estrutura que define diversos eventos como as \textit{sprints} que são ciclos de desenvolvimento com um tempo definido, geralmente de duas semanas, e a retrospectiva que é o momento em que a equipe discute o que foi bom ou ruim no ciclo (\textit{sprint}) que se passou. O \textit{Scrum} também define os membros da equipe e suas responsabilidades, como o PO, o \textit{Scrum Master} e a equipe de desenvolvimento \cite{scrum}. 

O \textit{Kanban} é uma estrutura que permite a visualização dos itens de trabalho que são organizados em um quadro que é dividido em \textit{To Do}, \textit{In Progress} e \textit{Done} \cite{kanbam}.

Para a avaliação do desempenho das SBCs selecionadas foram coletadas as informações de uso do processador, o uso de memória RAM e também a taxa de erros durante a transmissão de um pacote de dados.

\subsection{\textit{Single Board Computer} (SBC)}

\textit{Single Board Computer} (SBC) é um computador onde todos os componentes necessários estão em uma mesma placa de circuito impresso. Esse tipo de dispositivo é muito utilizado para fins educacionais, para desenvolvimento de sistemas, datacenters (centros  de  processamento  de  dados) e clusters portáteis. Alguns exemplos são o \textit{OrangePI}, \textit{RockPI}, \textit{BeagleBone} e \textit{RaspberryPI}, sendo este um dos mais populares \cite{SBC_edu}.

Os SBCs geralmente são de baixo custo, porém devido a escassez global de semicondutores reduziu a sua disponibilidade e por consequência levou ao aumento dos preços \cite{zeng_2022}. Principalmente do \textit{RaspberryPI} que passou de 45 dólares para 161 dólares, por esse motivo o SBC \textit{OrangePI} foi selecionado para o desenvolvimento deste trabalho.

\subsubsection{\textit{RaspberryPI}}

A \textit{Raspberry Pi Foundation} foi fundada em 2008 sediada no Reino Unido com o objetivo de promover o avanço na educação no campo da computação (\textit{\citeauthor{rasp}}, 2018).

O \textit{RaspberryPI} é um pequeno computador que traz consigo um processador na arquitetura ARM, a mesma tecnologia que se encontra em num smartphone (\textit{\citeauthor{rasp}}, 2018).

\begin{figure}[!htbp]
  \caption{\textit{Raspberry Pi} 3}
  \includegraphics[scale=0.4]{images/rasp.png}
  \legend{Fonte: \textit{\citeauthoronline{rasp}} (2018)}
  \label{figura:rasp}
\end{figure}

\subsubsection{\textit{OrangePI}}

O \textit{OrangePI} é um SBC \textit{open source} da \textit{Shenzhen Xunlong Software}. A arquitetura de seu processador é ARM e a plataforma suporta vários sistemas operacionais como Android e as várias distribuições de linux (\textit{\citeauthor{orangepi}}, 2023).

\begin{figure}[!htbp]
  \caption{\textit{Orange Pi} 3 LTS}
  \includegraphics[scale=0.35]{images/orange.png}
  \legend{Fonte: \textit{\citeauthoronline{orangepi}} (2023)}
  \label{figura:orange}
\end{figure}

\newpage

\subsubsection{\textit{BeagleBone Black}}

\textit{BeagleBone Black} é uma plataforma suportada pela comunidade que roda Linux para prototipagem rápida (\textit{\citeauthor{beaglebone}}, 2023).
Esta plataforma é utilizada pelo projeto OpenVLC. 

\begin{figure}[!htbp]
  \caption{\textit{BeagleBone Black}}
  \includegraphics[scale=0.58]{images/beaglebone.png}
  \legend{Fonte: \textit{\citeauthoronline{beaglebone}} (2023)}
  \label{figura:beagle}
\end{figure}

\subsection{Linux}

Linux é um sistema operacional de computadores, o autor \citeonline{negus} cita em seu livro Linux a Bíblia que este sistema é um exemplo de como projetos colaborativos podem ultrapassar o que empresas individuais podem fazer.

O Linux permite que os desenvolvedores alterem o sistema como quiserem ajudando a criar softwares para as suas necessidades, por esse motivo utilizaremos este sistema no desenvolvimento do trabalho.


  \bibliography{referencias}
  \nocite{OpenVLC, matheus2017comunicaccao}

\end{document}
